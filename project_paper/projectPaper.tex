\documentclass[12pt]{article}
\usepackage[utf8]{inputenc}
\usepackage{mathtools}
\usepackage{amsmath}
\usepackage{derivative}
\usepackage{amsthm}
\usepackage{amssymb}
\usepackage{listings}
\usepackage{hyperref}
\usepackage{tikz}
\usepackage{float}
\usepackage{float}
\usepackage{caption}
\usepackage{subcaption}
\usepackage{algorithm}
\usepackage{algpseudocode}
\usepackage{graphicx}
\usepackage{enumitem}
\usepackage[nottoc,numbib]{tocbibind}
\graphicspath{{./figures/}}
\usepackage[
backend=biber,
style=alphabetic,
sorting=ynt
]{biblatex}
\addbibresource{references.bib}

\newfloat{algorithm}{H}{lop} % float the alg environment

\newtheorem{definition}{Definition}[section]
\newtheorem{theorem}{Theorem}[section]
\newtheorem{corollary}{Corollary}[theorem]
\newtheorem{lemma}[theorem]{Lemma}

\title{Sending Secret Messages with Synchronized Chaotic Systems}
\author{Kevin Phan}
\date{May 12, 2023}

\begin{document}
    \maketitle

    \begin{abstract}
      This paper introduces an example of a synchronized chaotic system based on the Lorenz system. We will see how this can be applied to the field of communications as synchronized chaotic system can be used to send secret messages. Lastly, we see how resistant this method of encryption is to noise when transmitting the signal. 
    \end{abstract}

    \newpage

    \tableofcontents

    \newpage

    \section{Introduction}
      add introduction (do this last) \cite{cuomo1993}
    \section{Theory of Synchronized Chaotic Systems}
      % - for only the function that we are working with and show exponential convergence 
      % - give definition for synchronization (from that one paper)
      % - introduces system and where it comes from (from circuit implementation)
      A synchronized system is when two dynamical systems' trajectories are eventually identical as time $t \rightarrow \infty$. A definition of synchronization is given by He and Vaidya \cite{he_vaidya_1992}. 
      \begin{definition}
        Let $\dot{\mathbf{x}} = f(t,\mathbf{x})$ and $\dot{\mathbf{y}} = g(t,\mathbf{y})$ be two dynamical systems, where $t$ is time and $\mathbf{x},\mathbf{y} \in \mathbb{R}^n$. Let $\mathbf{x}(t;t_0,\mathbf{x}_0)$ and $\mathbf{y}(t;t_0,\mathbf{y}_0)$ be solutions to the dynamical systems respectively. We say that the two dynamical systems synchronize if there exists a subset of $\mathbb{R}^n$, denoted $D(t_0)$, such that $\mathbf{x}_0,\mathbf{y}_0 \in D(t_0)$ implies 
        \begin{equation*} 
          ||\mathbf{x}(t;t_0,\mathbf{x}_0) - \mathbf{y}(t;t_0,\mathbf{y}_0)|| \rightarrow 0 \ \text{as} \ t \rightarrow +\infty.
        \end{equation*}
        If the region of synchronization $D(t_0) = \mathbb{R}^n$, we say that the synchronization is global and otherwise, the synchronization is local. 
      \end{definition}
      Note synchronization does not depend on the initial conditions of the dynamical systems. 
      
      One example of a synchronized system is given by Cuomo and Oppenheim (1993) which was used in the application of communications and how to send secret messages: 
      \begin{equation}\label{eq:transmitter}
      \begin{aligned}
        \dot{x_T} &= \sigma (y_T-x_T), \\
        \dot{y_T} &= r  x_T - y_T - 20 (x_T   z_T),\\
        \dot{z_T} &= 5 x_T y_T - b  z_T,
      \end{aligned}
    \end{equation} 
    which is the transmitter's dynamical system and 
    \begin{equation}\label{eq:receiver}
      \begin{aligned}
        \dot{x_R} &= \sigma (y_R-x_R), \\
        \dot{y_R} &= r  x_T - y_R - 20 (x_T   z_R),\\
        \dot{z_R} &= 5 x_T y_R - b  z_R,
      \end{aligned}
    \end{equation}
    which is the receiver's dynamical system \cite{cuomo1993}. Notice that the only state variable of the transmitter's dynamical system that appears in the receiver's dynamical system is $x_T$. This means that only information that the receiver need to know to reconstruct the trajectory is data about $x_T$.  

    Cuomo and Oppenhiem proved that the system will synchronize using a Lyapunov function to show that the error asymptotically approaches the point $\mathbf{0} \in \mathbb{R}^3$ \cite{cuomo1993}. 
      \begin{theorem}
        The pair of dynamical systems given by equations \ref{eq:transmitter} and \ref{eq:receiver} are globally synchronized. 
      \end{theorem}
      \begin{proof} 
       Let $e_x = x_T - y_T$, $e_y = y_T - y_R$, and $e_z = z_T - z_R$. This gives us the dynamical system 
       \begin{equation}\label{eq:error}
        \begin{aligned}
         \dot{e_x} &= \sigma (e_y - e_x), \\ 
         \dot{e_y} &= -e_y - 20e_z x_T, \\
         \dot{e_z} &= 5e_y x_t - be_z,  
        \end{aligned} 
       \end{equation}
       which describes the error between each component of the trajectories given by transmitter and receiver's dynamical system. Let $\mathbf{e} = (e_x,e_y,e_z)^T$. To show synchronization, it is sufficient to show that for equation \ref{eq:error}, the fixed point $\mathbf{0}$ is asymptotically stable. By inspection, a fixed point of equation \ref{eq:error} is $\mathbf{0}$. 
       
       To show that it is asymptotically stable, we will use a Lyapunov function. Let $V(\mathbf{e}) = \frac{1}{2} \left( \frac{1}{\sigma} e_x^2 + e_y^2 + 4e_z^2 \right)$. By inspection, we see that $V(\mathbf{0}) = 0$ and $V(\mathbf{e}) > 0$ for all $\mathbf{e} \neq 0$. Taking the time derivative and substituting for $e_x,e_y,e_z$ using equation \ref{eq:error}, we get 
       \begin{align*} 
         \dot{V} &= \frac{1}{\sigma} e_x \dot{e_x} + e_y \dot{e_y} + 8 \dot{e_z} \\
         &= \frac{1}{\sigma} e_x (\sigma e_y - \sigma e_x) + e_y (-e_y - 20x_T e_z) + 4 e_z (5x_T e_y - be_z) \\ 
         &= -e_x^2 + e_x e_y - e_y^2 - 4be_z^2 \\
         &= - \left( e_x - \frac{1}{2}e_y \right)^2 - \frac{3}{4} e_y^2 - 4be_z^2,
       \end{align*}
       where the last step is completing the square. Thus, for all $\mathbf{e} \neq \mathbf{0}$, $\dot{V} < 0$. As such, $V$ is a Lyapunov function, so all trajectories of $V$ flow toward the fixed point $\mathbf{0}$. This means that the error goes to $0$ which prove that the pair of dynamical systems given by equations \ref{eq:transmitter} and \ref{eq:receiver} are globally synchronized.
      \end{proof}
      However, the definition of synchronization does not tell us how fast the pair of dynamical systems will synchronize. Fortunately, Cuomo, Oppenheim, and Strogatz (1993) proven that error converge exponentially by studying the error dynamics \cite{expProof}.\footnote{This is also given as Exercise 9.1 in Strogatz's Nonlinear Dynamics and Chaos \cite{strogatz2019nonlinear}.} 
      \begin{theorem}
        The error dynamics of equations \ref{eq:transmitter} and \ref{eq:receiver} given by equation \ref{eq:error} converge to $\mathbf{0}$ exponentially. In other words, $e_x, e_y, e_z = O(e^{-t})$. 
      \end{theorem}
      \begin{proof}
        Consider the function $V=\frac{1}{2}e_y^2 + 2e_z^2$. We wish to show that $\dot{V} \leq - k V$ where $k=\min\{2,2b\}$. Taking the time derivative and substituting for $e_y,e_z$ using equation \ref{eq:error}, we get
        \begin{align*}
          \dot{V} &= e_y \dot{e_y} + 4e_z \dot{e_z} \\ 
          &= e_y (-e_y - 20e_z x_T) + 4e_z (5e_y x_T - be_z) \\
          &= - (e_y^2 + 4be_z^2). 
        \end{align*}
        We see that $- (e_y^2 + 4be_z^2) \leq -k \left( \frac{1}{2} e_y^2 + 2e_z^2 \right)$ where $k=\min\{2,2b\}$. 

        Next, we establish that $e_2,e_3 = O(e^{-t})$. Integrating $\dot{V} = -kV$ with respect to $t$, we get $V(t) \leq V_0 e^{-kt}$ where $V_0$ is the initial condition of $V$. From this, we establish the inequalities 
        \begin{equation*}
          \frac{1}{2}e_y^2 \leq V(t) \leq V_0 e^{-kt}
        \end{equation*}
        and 
        \begin{equation*}
          2e_z^2 \leq V(t) \leq V_0 e^{-kt}. 
        \end{equation*}
        Solving for $e_y$ in the first inequality, we get 
        \begin{equation*}
          e_y \leq \sqrt{2V_0} e^{-\frac{kt}{2}}. 
        \end{equation*}
        Simiarly, solving for $e_z$ in the second inequality, we get 
        \begin{equation*} 
          e_z \leq \sqrt{\frac{V_0}{2}} e^{-\frac{kt}{2}}. 
        \end{equation*}
        This show that $e_y,e_z = O(e^{-t})$. 
        
        Lastly, we show that $e_1 = O(e^{-t})$. From equation \ref{eq:error}, $\dot{e_x} = \sigma (e_y - e_x)$. This can be rewritten as $\dot{e_1} + \sigma e_1 = \sigma e_2$. Since $e_y \leq \sqrt{2V_0} e^{-\frac{kt}{2}}$, we have 
        \begin{equation*} 
          \dot{e_1} + \sigma e_1 \leq \sqrt{2V_0} e^{-\frac{kt}{2}}. 
        \end{equation*}
        Multiplying by the integrating factor $e^{\sigma t}$, we get 
        \begin{equation*} 
          (e_1 e^{\sigma t})' \leq \sqrt{2V_0} e^{\left( \sigma - \frac{k}{2} \right)t} 
        \end{equation*}
        Integrating and solving for $e_1$, we get 
        \begin{equation*} 
          e_1 \leq \frac{\sqrt{2V_0}}{\sigma - \frac{k}{2}} e^{-\frac{kt}{2}}
        \end{equation*}
        This established that $e_1 = O(e^{-t})$. Therefore, we proved that the error dynamics given by equation \ref{eq:error} converge to $\mathbf{0}$ exponentially. 
      \end{proof}
      give numerical example of synchronization and exponential convergence (copy from presentation essentially)
    \section{Numerical Experiments}
    % based on circuit implementation \cite{cuomo1993}
    % summary of what we are doing 
    % can mention something about precision numbers 
    \subsection{Algorithm Implementation}
    % - numerical implementation of the algorithm and plots of convergence and noise and add pitfalls about machine epsilon 
    % - introduce it in the code format and mention particular findings that I found for this (numerical errors that can occur with rounding and what not)
    \subsection{Testing Algorithm Against Noise}
    % - testing it against Gaussian noise (add some noise and see how bad it can get)
    % - why it is not that good for sending secret messages but there are better method for doing so (binary messages which is more robust to sending secret messages)
    \section{Conclusion}
    % - brief summary and future work 
  \newpage    
    \printbibliography
    \addcontentsline{toc}{section}{References}
\end{document}