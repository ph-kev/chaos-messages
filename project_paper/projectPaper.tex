\documentclass[12pt]{article}
\usepackage[utf8]{inputenc}
\usepackage{mathtools}
\usepackage{amsmath}
\usepackage{derivative}
\usepackage{amsthm}
\usepackage{amssymb}
\usepackage{listings}
\usepackage{hyperref}
\usepackage{tikz}
\usepackage{float}
\usepackage{float}
\usepackage{caption}
\usepackage{subcaption}
\usepackage{algorithm}
\usepackage{algpseudocode}
\usepackage{graphicx}
\usepackage{enumitem}
\usepackage[nottoc,numbib]{tocbibind}
\graphicspath{{./figures/}}
\usepackage[
backend=biber,
style=alphabetic,
sorting=ynt
]{biblatex}
\addbibresource{references.bib}

\newfloat{algorithm}{H}{lop} % float the alg environment

\newtheorem{theorem}{Theorem}

\title{Sending Secret Messages with Synchronized Chaotic Systems}
\author{Kevin Phan}
\date{\today}

\begin{document}
    \maketitle

    \begin{abstract}
      This paper introduces the basics of synchronization of chaotic systems. We will see how this can be applied to the field of communications as synchronized chaotic system can be used to send secret messages. Lastly, we show how resistant this method of encryption is to noise. 
    \end{abstract}

    \newpage

    \tableofcontents

    \newpage

    \section{Introduction}
      add introduction \cite{cuomo1993}
    \section{Theory of Synchronized Chaotic Systems}
      - for only the function that we are working with and show exponential convergence 
      \begin{theorem}
        $E(t) \rightarrow 0$ 
      \end{theorem}
      \begin{theorem}
        exponential convergence 
      \end{theorem}
    \section{Numerical Experiments}
    based on circuit implementation \cite{cuomo1993}
    summary of what we are doing 
    can mention something about precision numbers 
    \subsection{Algorithm Implementation}
    algorithm and plots of convergence and noise and add pitfalls about machine epsilon 
    \subsection{Testing Algorithm Against Noise}
    testing it against Gaussian noise (add some noise and see how bad it can get)
    \section{Discussion}
    why it is not that good for sending secret messages but there are better method for doing so (binary messages which is more robust to sending secret messages)
  \newpage    
    \printbibliography
    \addcontentsline{toc}{section}{References}
\end{document}